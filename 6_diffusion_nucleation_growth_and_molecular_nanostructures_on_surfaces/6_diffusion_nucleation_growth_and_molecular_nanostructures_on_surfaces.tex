\documentclass[article,oneside]{memoir}

%cool packages
\usepackage[utf8]{inputenc}% That awesome utf8 characters
\usepackage[T1]{fontenc}% Those nice fonts
\usepackage{mathtools,amssymb,bm} %math
\usepackage{siunitx} % typesetting for units
\usepackage[draft,margin]{fixme} % for notes, remove draft if final version
\usepackage{graphicx} % for inserting figures
\usepackage{acronym} %for having acronyms
\usepackage{tikz} % for drawing stuff
\usepackage{pgfplots} %for plotting data or math functions
\usepackage{xcolor} % for having nicer colours
\usepackage{ifthen} % boolean checks

%penalties for having clubs or widows
\clubpenalty10000 
\widowpenalty10000

%cool memoir stuff
%The cool memoir stuff is wrapped in a check. This is done to see if the document class actually supports it.

\makeatletter%
\@ifclassloaded{memoir}%
  {%
\newsubfloat{figure} %allowing sub figures
\graphicspath{{./figures/}{../figures/}} % no need for those pesky folder extensions in the actual file
  }%
  {}%
\makeatother%

% new commands

\newcommand\numberthis{\addtocounter{equation}{1}\tag{\theequation}} %numbering equations in align* or aligned*

%\newcommand{colourscheme}[1]%
%{%
%        \ifthenelse{\eqal{#1}{1}}{%       
%                \definecolor{c1}{rgb}{0,0,0}
%                \definecolor{c2}{rgb}{0,0,0}
%                \definecolor{c3}{rgb}{0,0,0}
%                \definecolor{c4}{rgb}{0,0,0}
%                \definecolor{c5}{rgb}{0,0,0}
%                \definecolor{c6}{rgb}{0,0,0}
%                \definecolor{c7}{rgb}{0,0,0}
%                \definecolor{c8}{rgb}{0,0,0}
%                \definecolor{c9}{rgb}{0,0,0}
%                \definecolor{c10}{rgb}{0,0,0}
%                \definecolor{c11}{rgb}{0,0,0}
%                \definecolor{c12}{rgb}{0,0,0}
%                \definecolor{c13}{rgb}{0,0,0}
%                \definecolor{c14}{rgb}{0,0,0}
%                \definecolor{c15}{rgb}{0,0,0}
%                \definecolor{c16}{rgb}{0,0,0}
%        }%
%        {}%
%        \ifthenelse{\eqal{#1}{2}}{%       
%                \definecolor{c1}{rgb}{0,0,0}
%                \definecolor{c2}{rgb}{0,0,0}
%                \definecolor{c3}{rgb}{0,0,0}
%                \definecolor{c4}{rgb}{0,0,0}
%                \definecolor{c5}{rgb}{0,0,0}
%                \definecolor{c6}{rgb}{0,0,0}
%                \definecolor{c7}{rgb}{0,0,0}
%                \definecolor{c8}{rgb}{0,0,0}
%                \definecolor{c9}{rgb}{0,0,0}
%                \definecolor{c10}{rgb}{0,0,0}
%                \definecolor{c11}{rgb}{0,0,0}
%                \definecolor{c12}{rgb}{0,0,0}
%                \definecolor{c13}{rgb}{0,0,0}
%                \definecolor{c14}{rgb}{0,0,0}
%                \definecolor{c15}{rgb}{0,0,0}
%                \definecolor{c16}{rgb}{0,0,0}
%        }%
%        {}%
%} % this command creates some colours, based on the input. 

\definecolor{c1}{rgb}{0,0,0}
\definecolor{c2}{rgb}{0,0,0}
\definecolor{c3}{rgb}{0,0,0}
\definecolor{c4}{rgb}{0,0,0}
\definecolor{c5}{rgb}{0,0,0}
\definecolor{c6}{rgb}{0,0,0}
\definecolor{c7}{rgb}{0,0,0}
\definecolor{c8}{rgb}{0,0,0}
\definecolor{c9}{rgb}{0,0,0}
\definecolor{c10}{rgb}{0,0,0}
\definecolor{c11}{rgb}{0,0,0}
\definecolor{c12}{rgb}{0,0,0}
\definecolor{c13}{rgb}{0,0,0}
\definecolor{c14}{rgb}{0,0,0}
\definecolor{c15}{rgb}{0,0,0}
\definecolor{c16}{rgb}{0,0,0}

%%%             PACKAGE CUSTOMIZATION START             %%%


\usetikzlibrary{calc,patterns,shapes,arrows,positioning} %Cool libraries for drawing stuff

\sisetup{exponent-product=\cdot, output-product =\cdot,per-mode=fraction,range-phrase=--}%Nicer way to present units

%%%             PACKAGE CUSTOMIZATION END               %%%

\title{Diffusion, nucleation, growth and molecular nanostructures on surfaces}
\author{Thorbjørn Erik Køppen Christensen}
\begin{document}
\maketitle
\part{Diffusion nucleation and growth}
\chapter{Basic diffusion system}
\section{Random walk}
The idea is that adatoms can wander freely on a surface if the thermal energy is height enough. It does so by:
\begin{align*}
        \rangle \Delta r^2 \langle &= \nu a^2 t\\
        \intertext{$\rangle \Delta r^2 \langle$ is the mean square displacement, $\nu$ the jump rate (generally around \SI{10e13}{\per\second}), $a$ the jump distance and $t$ the time. This gives rice to the diffusion coefficient (diffusivity) or the time dependent ratio of the mean square displacement:}
        D&=\frac{\rangle\Delta r^2\langle}{zt}=\frac{\nu a^2}{z}\\
        \intertext{Where $z$ is the number of possible jumps (2 is 1d, 4 is 2d square, 6 is 2d hexagonal). The frequency can be found as a function of temperature and barrier:}
        \nu&=\nu_0e^{-\frac{E_{\textup{diff}}}{k_{b}T}}
        \intertext{$D$ might not be equal in every direction, a 1d system could have a small chance of jumping in another direction, if this is the case the diffusion is anisotropic, it could also be caused by the unit cell not being symmetric:}
        D(\phi)&=D_{x}\cos^2\phi+D_{y}\sin^2\phi
\end{align*}
\chapter{Ways the atoms move}
Atoms can move in different ways: 

Jumping over a neighbouring atom, while this might seem simple it's not.

Exchanging: Atoms can diffuse on a surface by changing place with the atom below it, pushing it in the diffusion direction.

\foldfig{width=\columnwidth}{diffusion_exchange_mechanism}
\section{Surface diffusion of clusters}
When islands grow new forms of diffusion, these are called cluster diffusions, and the idea is that the centre of mass for the cluster changes. They are:
\subsection{for individual atoms}
\begin{itemize}
        \item sequential displacement mechanism --- The atoms move one by one
        \item Edge diffusion --- Atoms move along the edge of the island
        \item Evaporation--condensation --- An atom evaporates from the island in one place and another condensates in another place. If the island is stable the processes will be in equilibrium
        \item Leapfrog --- An atom jumps over a series of atoms.
\end{itemize}

\foldfig{width=\columnwidth}{diffusion_individual_cluster_motion}
\subsection{for groups}
\begin{itemize}
        \item Gliding mechanism --- all atoms move simultaneously
        \item Shearing --- A subgroup of atoms glide, but not all atoms (and not a single atom)
        \item Reptation --- multiple shears gliding in a zigzag manner
        \item Dislocation --- rows are seperated by a misfit, the atoms will then move through shears or individual movements to eliminate the fault
\end{itemize}

\foldfig{width=\columnwidth}{diffusion_group_cluster_motion}

\chapter{Growing islands}
For a simple system with $i=1$ one has
\begin{align*}
        N&\sim\left( \frac{R\Theta}{\nu} \right)^\chi\\
        \chi&=\frac{i}{i+2}
\end{align*}
\section{modes}
Islands can grow in three modes:
\begin{itemize}
        \item \emph{Layer-by-layer} or \emph{Frank-van der Merve (FM)} one layer grows completely before the next one start on top. Caused by the film being stronger bound to the substrate than itself
        \item \emph{Layer plus island} (\emph{Stranski--Krastanow (SK)}) the first layer grows completely, then island can start growing in height
        \item \emph{Island} (\emph{Vollmer--Weber (VW)}) film is bound more to itself than the surface, and the islands can build height
\end{itemize}

\foldfig{width=\columnwidth}{nucleation_island_growth_modes}

The different modes can be understood by the surface tension $\gamma$
\begin{equation*}
        \gamma_{S}=\gamma_{S/F}+\gamma_{F}\cos\phi
\end{equation*}
$\gamma_{S}$ being the surface tension of the substrate, $\gamma_{F}$ the surface tension of the film, $\gamma_{S/F}$ the film--surface tension and $\phi$ the contact angle. The growth mode can be found by removing the cosine factor and checking which part is larger.
\section{Nucleation and growth}
When Islands grow they have a critical surface size $i$, $i$ is the size at which an island breaks. So $i+1$ is the size at which the island is stable. There are multiple important factors for how the islands grow. some are $R$ the rate at which atoms arrive from gas phase, $E_{\textup{ads}}$ the energy binding the adatom to the surface, putting $n_{1}$ adatoms on the surface with $n_{_0}$ sites per unit area, the diffusion coeficcient 
\begin{align*}
        D&=\frac{\nu}{4n_{0}}e^{-\frac{E_{\textup{diff}}}{k_{B}T}}\\
        \intertext{The adatom can then reevaporate with the lifetime}
        \tau_{\textup{ads}}&=\frac{1}{\nu}e^{\frac{E_{\textup{ads}}}{k_{B}T}}\\
\end{align*}

Adatoms might also be captured by existing islands, they can also combine to form a new cluster

\foldfig{width=\columnwidth}{nucleation_differential_equation_origin}



\subsection{Book mathematics}
For a cluster of size $j$ there are four important processes:
\begin{itemize}
        \item A new cluster size $j$ is formed every time a cluster size $j-1$ get's an atom. The flux is $\sigma_{j-1}Dn_{j-1}n_{1}$
        \item The detachment of an atom from a cluster size $j+1$, the flux is $\partial_{j+1}n_{j+1}$ also creates a new group of size $j$
        \item A group of size $j$ is lost when it losses an atom at flux $\partial_jn_j$
        \item A group of size $j$ is lost when it gains an atom, flux $\sigma_{j}Dn_jn_1$
\end{itemize}
$n_1$ is the density of adatoms, $D$ the diffusion coefficient, $\sigma$ the capture number (capability of cluster to capture atom) and $\partial_{j+1}= De^{-\frac{\Delta E_{j}^{j+1}}{k_{B}T}}$ is the decay rate of an island, $\Delta E_{j}^{j+1}$ is the energy difference between the two island sizes.
$n_{x}$ is now the number density of stable islands ($j>i$)

\begin{align*}
        \frac{dn_1}{dt}&=\underbrace{R}_{\textup{desposition}}-\underbrace{\frac{n_1}{\tau_{\textup{ads}}}}_{\textup{dissociation}}+\left( \underbrace{2\partial_{2}n_{2}}_{\textup{nucleation}}+\underbrace{\sum_{j=3}^{i}\partial_{j}n_j}_{\textup{growth}}-\underbrace{2\sigma_1Dn_1^2}_{\textup{denucleation}}-\underbrace{n_1\sum_{j=2}^i\sigma_jDn_j}_{\textup{degrowth}} \right)-\underbrace{n_1\sigma_xDn_x}_{\textup{growth stable}}\\
        \frac{dn_j}{dt}&=\underbrace{n_1\sigma_{j-1}Dn_{j-1}}_{\textup{growth}}-\underbrace{\partial_{j}n_j}_{\textup{degrowth}}+\underbrace{\partial_{j+1}n_{j+1}}_{\textup{degrowth}}-\underbrace{n_1\sigma_jDn_j}_{\textup{growth}}\\
        \frac{dn_x}{dt}&=\underbrace{n_1\sigma_1Dn_1}_{\textup{growth}}\\
        \intertext{This splits the system into 4 stages: ``low coverage'' (adatom density higher than island density), ``intermediate coverage'' (more islands than adatoms), ``aggregation regime'' (collecting the last adatoms) and the ``coalescence and percolation regime'' (Islands melting together). The number of island grown (normalized to sites) are:}
        \frac{n_{x}}{n_{0}}&=\eta(\Theta,i)\left( \frac{R}{Dn_{0}^{2}} \right)^{\chi} e^{\frac{E_{i}}{(i+2)k_{B}T}}\qquad\textup{with no reevaporation}\\
        \frac{n_{x}}{n_{0}}&= \eta(\Theta,i) \left( \frac{4R}{\nu_{0}n_{0}} \right)^{\chi} e^{\frac{\chi\left(E_{diff}+\frac{E_{i}}{i}\right)}{k_{B}T}}\\
        \chi_{2d}&=\frac{i}{i+2}\\
        \chi_{1d}&=\frac{i}{2(i+1)}\\
        \intertext{Here $E_{i}$ is the binding energy of the critical cluster, $\eta$ an exponential factor (\numrange{e-2}{e+1}). The second formulae takes temperature dependence of diffusion coefficient into account.}
\end{align*}
\subsection{Trolle mathematics}
!!!!OBS!!!!

\emph{Trolle did not care for adatoms leaving the surface in his 2017 lecture, also it's in general more lazy math than the one in the book}
\begin{align*}
        \frac{dN_{1}}{dt}&=R(1-\Theta)-\frac{F_{x}N_1}{\tau}-(i+1)\frac{dN_{x}}{dt}\\
        \frac{dN_{x}}{dt}&=\frac{F_{i}N_{1}}{\tau}\\
        N_{x}&\sim \eta(\Theta,i) \left( \frac{R}{\nu} \right)^{\chi} e^{\frac{\chi\left(E_{diff}+\frac{E_{i}}{i}\right)}{k_{B}T}}
\end{align*}

$N_1$: density of adatoms

$N_{x}$: density of stable islands

$F_i$: fraction of critical clusters

$F_{x}$: fraction of stable islands

$\tau$ average lifetime before adatom is lost to nucleation or growth.

\foldfig{width=\columnwidth}{nucleation_rate_vs_temp}

\section{Island size distribution and coarsing phenomena}

Two island shape types exist \emph{ramified} (fractal like) (low T $\rightarrow$ low edge diffusion) and \emph{Compact} (having straight lines for edges).
\subsection{island size}

\begin{align*}
        N&=\sum_{s>i}N_s\qquad \textup{noncritical islands}\\
        \Theta&=\sum_{s\geq 1}sN_{s}\\
        \langle s\rangle &= \frac{\sum_{s>i}sN_s}{\sum_{s>i}N_s}\\
        &=\frac{\Theta-\sum_{s\leq i}SN_{s}}{N}\\
        \intertext{if $i$ is small:}
        \langle s\rangle &= \frac{\Theta-N_1}{N}\approx \frac{\Theta}{N}\\
        f_{i}\left( \frac{s}{\langle s\rangle} \right)&=\frac{N_s(s_{av})^{2}}{\Theta}
\end{align*}

\foldfig{width=\columnwidth}{nucleation_capture_zone}
Screenshot from 2018-06-17 02-18-35
\subsection{Ostwald Ripening and coalscence}

coalscence is the merging of islands when they physically come into contact with one another, the shape depends on the edge mobility. (dynamic coalescence or Smoluschowsi ripening)
\foldfig{width=\columnwidth}{nucleation_coalscence}

Ostwald ripening is a diffusion process, where the equilibrium is in favor of the larger island, so the larger island consumes the smaller one without them touching.

$$\mu(r)=\Omega\frac{\gamma}{r}$$ is the Gibbs--Thompson relation (chemical potential for circular island), $\gamma$ the step line tension, $\Omega$ the area occupied by one atom, this makses the adatom pressure higher on smaller atoms, and thus Ostwald ripening is explained.
\foldfig{width=\columnwidth}{nucleation_oswald_ripening}

\part{Molecular nanostructures on surfaces}

\chapter{Self assembly}
``Molecular self-assembly is the spontanous association of
molecules under equilibrium conditions into stable, structurally
well defined aggregates joined by non-covalent bonds.''
\begin{equation*}
        E_b> E_{\textup{intermidiate}}\geq E_{\textup{kin}} > E_{\textup{dessorption}}
\end{equation*}


\section{Covalent  vs metal vs hydrogen}
Hydrogen bonds are lower en energy, so it's easier for them to restructure into something organized, metal bonds are semi organized, and covalent bonds are completely random.



\end{document}


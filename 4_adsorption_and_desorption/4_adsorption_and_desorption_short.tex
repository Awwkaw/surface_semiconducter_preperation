\documentclass[colspace=0.5cm,blockverticalspace=1cm]{tikzposter}
\usepackage[utf8]{inputenc}
\usepackage[T1]{fontenc}
\usepackage{mathtools,amssymb,bm}
\usepackage{graphicx}
\graphicspath{{./figures/}{../figures/}{../../figures/}{../figures/1/}{../figures/2/}{../figures/3/}{../figures/4/}{../figures/5/}{../figures/6/}{../figures/7/}{../figures/8/}{../figures/9/}{../figures/10/}} % no need for those pesky folder extensions in the actual file
\usetheme{basic}
\defineblockstyle{Justtitle}{
        }{
        \ifBlockHasTitle
                \draw[color=framecolor, fill=blocktitlebgcolor,
                rounded corners=\blockroundedcorners] (blocktitle.south west)
                rectangle (blocktitle.north east);
        \fi
}
\newcommand{\titleblock}[1]{\useblockstyle{Justtitle}\block{#1}{}\useblockstyle{Default}}
\newcommand{\foldfig}[2]{\includegraphics[#1]{#2/#2}}

\title{\centering{Adsorption and desorption}}
\author{Thorbjørn Erik Køppen Christensen}


\begin{document}
\maketitle
\begin{columns}
        \column{0.5}
        \titleblock{Physisorption}
        \begin{subcolumns}
                \subcolumn{0.48}
                \block{Principle}{\foldfig{width=0.94\subcolwidth}{physisorbtion_vs_van_der_waals}}
                \subcolumn{0.48}
                \block{Math}{
                        Normal vdW:
                        \begin{align*}
                                E_1&\propto r^{-3}\\
                                p_2&\propto \alpha p_{1}r^{-3}\\
                                U&\propto p_{2}E_{1}\propto r^{-6}\\
                                \intertext{In physisorption the second dipole is an image charge:}
                                p_2& =p_1\\
                                U&\propto p_2 E_{1}\propto r^{-3}
                        \end{align*}
                        The range is several \si{\angstrom}, the energyrange is \SIrange{10}{100}{\milli\eV}}
        \end{subcolumns}
        \titleblock{Chemisorption}
        \begin{subcolumns}
                \subcolumn{0.31}               
                \block{d band scheme}{\foldfig{width=0.87\subcolwidth}{chemisorbtion_d_band_schematic}}
                \subcolumn{0.31}               
                \block{metal scheme}{\foldfig{width=0.87\subcolwidth}{chemisorbtion_simple_metal_schematic}}
                \subcolumn{0.31}               
                \block{dissociative adsorption}{\foldfig{width=0.87\subcolwidth}{dissociative_adsorption}}
        \end{subcolumns}
                \block{Newns--Anderson model math}{
\begin{align*}
        \psi&=c_m\psi_m+c_a\psi_a\\
        \intertext{where $m$ is the metal and $a$ is the adsorbate. The system then has the energies:}
        E_m&=\int\psi_{m}^*H\psi_{m}\,d\mathbf{r}\\
        E_a&=\int\psi_{a}^*H\psi_{a}\,d\mathbf{r}\\
        -V=\int\psi_{a}^*H\psi_{m}\,d\mathbf{r}&=\int\psi_{m}^*H\psi_{a}\,d\mathbf{r}\\
        \intertext{And the overlap}
        S&=\int\psi_{m}^*\psi_{a}\,d\mathbf{r}\\
        H\psi&=E\psi\\
        \int \psi_{a}^*H\psi=-c_mV+c_aE_a&=E_a(Sc_m+c_a)\\
        \int \psi_{m}^*H\psi=-c_aV+c_mE_m&=E_m(Sc_a+c_m)\\
        \intertext{Or in matrix form:}
        \left[\begin{array}{cc}
                E-E_a & V-ES\\
                V-SE & E-E_m
        \end{array}\right] &= 0\\
        \intertext{For simplicity $S=0$ can be assumed}
        E_{1,2}&= \frac{E_a+E_m}{2}\pm \sqrt{\left(\frac{E_a-E_m}{2}\right)^2+V^2}\\
        E_{1,2}&= \bar{E} \pm\Delta\\
        c_a&=c_m \frac{E_1-E_m}{V}\qquad\textup{For the lower state}\\
        c_a&=c_m \frac{E_2-E_m}{V}\qquad\textup{For the upper state}\\
\end{align*}}

\column{0.48}
\titleblock{Kinetics --- The Langmuir model}
        \begin{subcolumns}
                \subcolumn{0.5}
                \titleblock{Adsorption}
%                \begin{subcolumns}
%                        \subcolumns{0.5} 
                        \block{Experimental results}{\foldfig{width=0.94\subcolwidth}{langmuir_experimental}}
%                        \subcolumns{0.5} 

%                \end{subcolumns}
%
                \subcolumn{0.48}
%
                \titleblock{Desorption}
%                \begin{subcolumns}
%                        \subcolumns{0.5} 
                        \block{TDS experiment}{\foldfig{width=0.94\subcolwidth}{temperature_programmed_desorption}}
%                        \subcolumns{0.5} 

                \end{subcolumns}
\block{Langmuir math}{
\begin{align*}
        \frac{d\Theta}{dt}=S \frac{dN}{dt}&=S \frac{P}{\sqrt{2\pi Mk_{b}T} }\\
        \intertext{where, $N$ is the number of sites ($N_{0}$ would be the number of sites on a clean surface), $S$ is the sticking coefficient:}
        S=c(1-\Theta)^ne^{\frac{E_{a}}{k_{b}T}}&=S_{0}(1-\Theta)^n\\
        \intertext{$S_{0}$ is the sticking coefficient on a clean surface. $c$ is the fraction of incoming molecules adsorbed on a clean surface. $(1-\Theta)^n$ takes care of the fact that the adsorption probability changes with coverage. The $^n$ describes the order; at zeroth order new adsorbants can sit on top of old adsorbants, at first order the adsorption is associative, at second order the adsorption is dissociative. The last factor is the energy requirement  for the process, in total this gives:}
        \frac{d\Theta}{dt}&= \frac{P}{\sqrt{2\pi Mk_{b}T} }c(1-\Theta)^ne^{\frac{E_{a}}{k_{b}T}}
\end{align*}
$c$, $E_{a}$ and $n$ are unknown.

This model is often too simple, as it doesn't mind physisorption. Physisorption can act as a precursor to chemisorption and thus making binding more likely at lower coverages and less likely at higher coverages.

}                        
\block{Langmuir desorption and Clausius–Claperyon formulæ}{
\begin{equation*}
        T=T_0+\beta t
\end{equation*}
\begin{equation*}
        -\frac{d\Theta}{dt}\propto P_{\textup{partial}}
\end{equation*}
\begin{align*}
        -\frac{d\Theta}{dt}&=\nu\Theta^ne^{-\frac{E_{d}}{k_{b}T}}\\
        \intertext{maximum at:}
        \frac{d^2\Theta}{dt^2}&=0\\
        &\Downarrow\qquad n=1\\
        E_{d}&=kT_{m}\ln\left(\frac{kT_m^2\nu}{E_d\beta}\right)
\end{align*}
\begin{align*}
         \nu\Theta^ne^{-\frac{E_{d}}{k_{b}T}}  &=      \frac{P}{\sqrt{2\pi Mk_{b}T} }c(1-\Theta)^ne^{\frac{E_{a}}{k_{b}T}}  \\
        \intertext{Or:}
        P&=  \frac{\sqrt{2\pi Mk_{b}T}}{c}\nu_n\left(\frac{\Theta}{1-\Theta}\right)^ne^{-\frac{E_{d}-E_{a}}{k_{b}T}}\\
        E_{d}-E_{a}&=-k\left(\frac{\partial \ln(P)}{\partial \frac{1}{T}}\right)\bigg\vert_{\Theta}
\end{align*}
}
        \end{columns}

\end{document}

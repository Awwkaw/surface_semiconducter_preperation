\documentclass[article,oneside]{memoir}

%cool packages
\usepackage[utf8]{inputenc}% That awesome utf8 characters
\usepackage[T1]{fontenc}% Those nice fonts
\usepackage{mathtools,amssymb,bm} %math
\usepackage{siunitx} % typesetting for units
\usepackage[draft,margin]{fixme} % for notes, remove draft if final version
\usepackage{graphicx} % for inserting figures
\usepackage{acronym} %for having acronyms
\usepackage{tikz} % for drawing stuff
\usepackage{pgfplots} %for plotting data or math functions
\usepackage{xcolor} % for having nicer colours
\usepackage{ifthen} % boolean checks

%penalties for having clubs or widows
\clubpenalty10000 
\widowpenalty10000

%cool memoir stuff
%The cool memoir stuff is wrapped in a check. This is done to see if the document class actually supports it.

\makeatletter%
\@ifclassloaded{memoir}%
  {%
\newsubfloat{figure} %allowing sub figures
\graphicspath{{./figures/}{../figures/}} % no need for those pesky folder extensions in the actual file
  }%
  {}%
\makeatother%

% new commands

\newcommand\numberthis{\addtocounter{equation}{1}\tag{\theequation}} %numbering equations in align* or aligned*

%\newcommand{colourscheme}[1]%
%{%
%        \ifthenelse{\eqal{#1}{1}}{%       
%                \definecolor{c1}{rgb}{0,0,0}
%                \definecolor{c2}{rgb}{0,0,0}
%                \definecolor{c3}{rgb}{0,0,0}
%                \definecolor{c4}{rgb}{0,0,0}
%                \definecolor{c5}{rgb}{0,0,0}
%                \definecolor{c6}{rgb}{0,0,0}
%                \definecolor{c7}{rgb}{0,0,0}
%                \definecolor{c8}{rgb}{0,0,0}
%                \definecolor{c9}{rgb}{0,0,0}
%                \definecolor{c10}{rgb}{0,0,0}
%                \definecolor{c11}{rgb}{0,0,0}
%                \definecolor{c12}{rgb}{0,0,0}
%                \definecolor{c13}{rgb}{0,0,0}
%                \definecolor{c14}{rgb}{0,0,0}
%                \definecolor{c15}{rgb}{0,0,0}
%                \definecolor{c16}{rgb}{0,0,0}
%        }%
%        {}%
%        \ifthenelse{\eqal{#1}{2}}{%       
%                \definecolor{c1}{rgb}{0,0,0}
%                \definecolor{c2}{rgb}{0,0,0}
%                \definecolor{c3}{rgb}{0,0,0}
%                \definecolor{c4}{rgb}{0,0,0}
%                \definecolor{c5}{rgb}{0,0,0}
%                \definecolor{c6}{rgb}{0,0,0}
%                \definecolor{c7}{rgb}{0,0,0}
%                \definecolor{c8}{rgb}{0,0,0}
%                \definecolor{c9}{rgb}{0,0,0}
%                \definecolor{c10}{rgb}{0,0,0}
%                \definecolor{c11}{rgb}{0,0,0}
%                \definecolor{c12}{rgb}{0,0,0}
%                \definecolor{c13}{rgb}{0,0,0}
%                \definecolor{c14}{rgb}{0,0,0}
%                \definecolor{c15}{rgb}{0,0,0}
%                \definecolor{c16}{rgb}{0,0,0}
%        }%
%        {}%
%} % this command creates some colours, based on the input. 

\definecolor{c1}{rgb}{0,0,0}
\definecolor{c2}{rgb}{0,0,0}
\definecolor{c3}{rgb}{0,0,0}
\definecolor{c4}{rgb}{0,0,0}
\definecolor{c5}{rgb}{0,0,0}
\definecolor{c6}{rgb}{0,0,0}
\definecolor{c7}{rgb}{0,0,0}
\definecolor{c8}{rgb}{0,0,0}
\definecolor{c9}{rgb}{0,0,0}
\definecolor{c10}{rgb}{0,0,0}
\definecolor{c11}{rgb}{0,0,0}
\definecolor{c12}{rgb}{0,0,0}
\definecolor{c13}{rgb}{0,0,0}
\definecolor{c14}{rgb}{0,0,0}
\definecolor{c15}{rgb}{0,0,0}
\definecolor{c16}{rgb}{0,0,0}

%%%             PACKAGE CUSTOMIZATION START             %%%


\usetikzlibrary{calc,patterns,shapes,arrows,positioning} %Cool libraries for drawing stuff

\sisetup{exponent-product=\cdot, output-product =\cdot,per-mode=fraction,range-phrase=--}%Nicer way to present units

%%%             PACKAGE CUSTOMIZATION END               %%%


%\colorscheme{1}
%\depth{1}
\begin{document}
\chapter{X-ray Photoelectron Spectroscopy (XPS)}
\foldfig{}{xps_scattering}

In the photoemission process the energy is conserved:
\begin{align*}
        E_{\textup{kin}}&=h\nu -E_{\textup{bin}} -\Phi\\
        E_{\textup{bin}}&=h\nu - E_{\textup{kin}} -\Phi
\end{align*}
$E_{\textup{kin}}$ is the kinetic energy of the electron. $h\nu$ the energy of the ingoing electron, $E_{\textup{bin}}$ the binding energy of the electron and $\Phi$ is the work function of the surface.
In XPS photons go in and electrons come out hence photo-electron. The exact binding energy depends heavily on the chemical environment, and this can make a chemical shift of \SIrange{1}{8}{\eV} the with and intensity 
$$I(E_{\textup{kin}})\approx\textup{DOS}(E_{\textup{bin}}=h\nu-E_{\textup{kin}}-\Phi)$$

The surface core level shift (SCLS) is due to the surface having a different chemical environment (having lost some (3) neighbours) There can even be multiple shifts for each layer, due to reconstruction and relaxation having effects longer into the material.

\section{Koopman's theorem}
This states that the measured binding energies are calculated one electron eigenvalues of the system. This changes the formula:


\begin{equation*}
        E_{\textup{kin}}=h\nu-(E_{\textup{bin}}-E_{\textup{r}})-\Phi
\end{equation*}
Where $(E_{\textup{bin}}-E_{\textup{r}})$ is the new ``apparent'' binding energy.
This is a final state effect, and it moves the pecieved energy up compared to the real energy(Adiabatic case). Furthere more plasmons will generate satellite peaks at a lower energy (sudden effect) these are only in metals and happen because relaxatino energy can go into low energy excitations and create plasmons. This phenomenon is called Doniach--Sunjic lines.

\section{Cross section}
The XPS peak intensity is, for the element $x$:
\begin{align*}
        I_{x}&=\sigma_{x}(E)N_{x}\lambda(E)AD(E)T(E)\\
        \intertext{$\sigma_{x}$ is the element specific cross--section, $N_{x}$ the concentration of the element, $\lambda$ is the inelastic electron mean free path, $A$ is the sample area, $D$ the detector efficiency and $T$ the analyzer transmission efficiency.}
        \frac{I_{A}}{I_{B}}&= \frac{\sigma_{A}(E)N_{A}\lambda(E)AD(E)T(E)}{\sigma_{B}(E)N_{B}\lambda(E)AD(E)T(E)}\\
        \frac{N_{A}}{N_{B}}&\approx \frac{I_{A}}{I_{B}} \times \frac{\sigma_{B}\lambda_{B}}{\sigma_{A}\lambda_{A}}\\
        \intertext{using Fermis golden rule, an expression for the differential cross section for phootoemission into the element of solid angode $d\Omega$}
        \frac{d\sigma}{d\Omega}&\propto\big\lvert\int\psi_{f}(R)\mathbf{H}'\psi_{i}(r)\,dr\big\rvert^2\partial(E_{f}-E_{i}-h\nu).\\
        \intertext{Where $\mathbf{H}'$ is the pertubation hamiltonian, it ccan be found by replacingthe momentum operator $\mathbf{p}$ by $\mathbf{p}-e\mathbf{A}$, $\mathbf{A}$ being the vector potential:}
        \mathbf{H}_0&=-\frac{e}{2m_{e}}(\mathbf{p}\cdot \mathbf{A}+\mathbf{A}\cdot \mathbf{p})-e\Phi+\frac{e^2}{2m_{e}}\lvert A\rvert^2\\
        \intertext{$\lvert A\rvert$ is small, so the last term can be ignored, the pertubatyon is then:}
        \mathbf{H}'&=-\frac{e}{2m_{e}}(\mathbf{p}\cdot \mathbf{A}+\mathbf{A}\cdot \mathbf{p})\\
        \intertext{using $[ \mathbf{p},\mathbf{A}]=-ih\nabla \mathbf{A}$}
        \mathbf{H}'&=-\frac{e}{2m_{e}}(2\mathbf{p}\cdot \mathbf{A}-ih\nabla\mathbf{A})\\
        \intertext{For low energy light, the vector potential is constant, so:}
        \frac{d\sigma}{d\Omega}&\propto\big\lvert\mathbf{A}_0\int\psi_{f}(R)\mathbf{p}\psi_{i}(r)\,dr\big\rvert^2\partial(E_{f}-E_{i}-h\nu).
        \intertext{This can also be written:}
        \frac{d\sigma}{d\Omega}&\propto\big\lvert\mathbf{A}_0\int\psi_{f}(R)\mathbf{r}\psi_{i}(r)\,dr\big\rvert^2\partial(E_{f}-E_{i}-h\nu).
\end{align*}
This cross section can be split into an angular part that gives the selection rules $l'=l\pm1$ and $m'=m,m\pm1$, and a radial part giving the cross section. 
The selection rules means that an s state electron must come out with p symmetry, also the polarization matters $\mathbf{A}_0$ is the polarization.

\section{Spin orbit coupling}
Spin orbit coupling means that the J quantum number is $J=L+S$, and it can be either parralel ($j=l+\frac{1}{2}$, $2j+1$ fold) or anti-parallel ($j=l-\frac{1}{2}$, $2j+1$ fold). Thus it only occurs for $l=1,2,3$

\foldfig{}{xps_j_splitting}

\chapter{Auger Electron Spectroscopy (AES)}
\foldfig{}{auger_electron_scheme}

Auger electrons are generated by shooting an electron at the surface, and knocking out a core electron. After this has happened another electron from an outer shell can then take it's place, in doing so releasing energy. The released energy can either be send out a photon, or be used to emit an electron from the original state of the electron.

Say the core electron has energy $E_A$, the electron jumping into the core has energy $E_{B}$ and the electron being shot out is in the energy state $E_{C}$.
This all assumes that the element is in state $Z$ for all events. 
\begin{align*}
        E_{\textup{kin}}&=E_{A}^{Z}-E_{B}^{Z}-E_{C}^{Z}-\Phi\\
        \intertext{But there is a core hole after electron A has left:}
        E_{\textup{kin}}&=E_{A}^{Z}-E_{B}^{Z+1}-E_{C}^{Z+1}-\Phi\\
        \intertext{But the atom is not $Z+1$:}
        E_{\textup{kin}}&=E_{A}^{Z}-\frac{1}{2}\big(E_{B}^{Z+1}+E_{B}^{Z}\big)-\frac{1}{2}\big(E_{C}^{Z+1}+E_{C}^{Z}\big)-\Phi
\end{align*}
is a good compromise, it's known as the $Z+1$ approximation

The core hole can be generated by x-rays and as such auger electrons will appear in XPS spectrum's, $E_{\textup{kin}}^{\textup{XPS}}$ depends on the input energy, $E_{\textup{kin}}^{\textup{AES}}$ does not, so this can be used to differntiate them.
\section{Auger nomeclature}
The nomeclature is (example) KLL, where the first letter is the initial hole, second letter is the shell of the falling electron, and the third letter is the shell of the emitted electron
\chapter{Extended X-ray Absorption Fine Structure (EXAFS)}
\foldfig{}{exafs_occilations}

Possible only with synchrotron radiation!
\section{EXEFAS}
There are oscillations in the absorption spectrum. These are due to the transfer from one atom to the next, so the same atom doesn't absorb and emit the electron.
\begin{align*}
        \chi(k)&=\frac{\sigma(k)-\sigma_{0}(k)}{\sigma_{0}(k)}\\
        \chi{}(k)&=-k^{-1}\sum{}_{i}A_{i}(k)\sin[2kR_{i}+\phi_{i}(180^\circ{},k)]\\
        \intertext{here the sum is over the shells, the $2kR_{i}$ term is the actual modulation and  the $\phi_{i}$ term is the energy dependant phase shift.}
        A_{i}&=\frac{N_i}{R_{i^2}}\lvert f_{i}(180^\circ,k)\rvert W(T,K)e^-\frac{2R_{i}}{\lambda}
\end{align*}

where $N_{i}$ is the number of atoms in sheel i, $R_{i}$ the radial dependance, making only the first few shells important. $f_{i}$ the scattering function (atom dependant), $W$ the Debye--Waller factor and the exponential tirm arises form it being inelastic scattering.
\section{Surface EXAFS (SEXAFS)}
Is the surface version of EXAFS, it gives the bond length but not absorption site. However, by polarizing the different neighbours can be seen one by one. There are however problems: Auger electrons will be generated and EXAFS only raises \SI{3}{\percent} 
\section{Near-edge EXAFS (NEXAFS)}
These are for the oscillations very close to the K-edge of the system. Strong peaks will be seen when enough energy is given to raise the $s1$ electron to the LUMO, LUMO+1, \dots.
\chapter{Photoelectron diffraction (PhD)}
PhD is a lot like XPS, but stems from the fact that the photoelectrons can come from deeper levels, and then create diffraction patterns with teach other. PhD makes \SIrange{30}{50}{\percent} of the modulations.
\begin{align*}
        I(\mathbf{R})&\propto{}\lvert{}\psi{}_{0}(\mathbf{R}+\sum{}_{j}\psi{}_{sj}\mathbf{R}\rvert{}^2\\
        I(k)&\propto\big\lvert\cos\Theta_{k}+\sum_j \frac{\cos\Theta_k}{r_{j}}\lvert f(\Theta_j,k)\rvert e^{ikr_{j}(1-\cos\Theta_j)+\phi(\Theta_j,k)}\big\rvert^2
\end{align*}
Where the first term is the direct wave and the second is the sum over all scatters. The $\phi$ term is the phase.
\section{Angle scan}
Here the angle varied, but the x-ray energy is constant, so it can be used with laboratory sources. The diffraction pattern can tell something about the surface structure.

\section{Energy scan}
Here the energy is varied, but angle kept constant, so a synchrotron is needed. It turns out that the spectrum will be modulated due to the difference in energy.
\begin{equation*}
        \chi(E)= \frac{I(E)-I_{0}(E)}{I_{0}(E)}
\end{equation*}
This experiment doesn't need long order to work, and XPS is underlying, so the same structural information is gathered

\chapter{Generating X-rays}

\foldfig{}{x-ray_generation}

\chapter{electron mass analyser}
\foldfig{}{electron_mass_analyzer}
\end{document}

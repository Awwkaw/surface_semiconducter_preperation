
%cool packages
\usepackage[utf8]{inputenc}% That awesome utf8 characters
\usepackage[T1]{fontenc}% Those nice fonts
\usepackage{mathtools,amssymb,bm} %math
\usepackage{siunitx} % typesetting for units
\usepackage[draft,margin]{fixme} % for notes, remove draft if final version
\usepackage{graphicx} % for inserting figures
\usepackage{acronym} %for having acronyms
\usepackage{tikz} % for drawing stuff
\usepackage{pgfplots} %for plotting data or math functions
\usepackage{xcolor} % for having nicer colours
\usepackage{ifthen} % boolean checks

%penalties for having clubs or widows
\clubpenalty10000 
\widowpenalty10000

%cool memoir stuff
%The cool memoir stuff is wrapped in a check. This is done to see if the document class actually supports it.

\makeatletter%
\@ifclassloaded{memoir}%
  {%
\newsubfloat{figure} %allowing sub figures
\graphicspath{{./figures/}{../figures/}} % no need for those pesky folder extensions in the actual file
  }%
  {}%
\makeatother%

% new commands

\newcommand\numberthis{\addtocounter{equation}{1}\tag{\theequation}} %numbering equations in align* or aligned*

\newcommand{colourscheme}[1]%
{%
        \ifthenelse{\eqal{#1}{1}}{%       
                \definecolor{c1}{rgb}{0,0,0}
                \definecolor{c2}{rgb}{0,0,0}
                \definecolor{c3}{rgb}{0,0,0}
                \definecolor{c4}{rgb}{0,0,0}
                \definecolor{c5}{rgb}{0,0,0}
                \definecolor{c6}{rgb}{0,0,0}
                \definecolor{c7}{rgb}{0,0,0}
                \definecolor{c8}{rgb}{0,0,0}
                \definecolor{c9}{rgb}{0,0,0}
                \definecolor{c10}{rgb}{0,0,0}
                \definecolor{c11}{rgb}{0,0,0}
                \definecolor{c12}{rgb}{0,0,0}
                \definecolor{c13}{rgb}{0,0,0}
                \definecolor{c14}{rgb}{0,0,0}
                \definecolor{c15}{rgb}{0,0,0}
                \definecolor{c16}{rgb}{0,0,0}
        }%
        {}
        \ifthenelse{\eqal{#1}{2}}{%       
                \definecolor{c1}{rgb}{0,0,0}
                \definecolor{c2}{rgb}{0,0,0}
                \definecolor{c3}{rgb}{0,0,0}
                \definecolor{c4}{rgb}{0,0,0}
                \definecolor{c5}{rgb}{0,0,0}
                \definecolor{c6}{rgb}{0,0,0}
                \definecolor{c7}{rgb}{0,0,0}
                \definecolor{c8}{rgb}{0,0,0}
                \definecolor{c9}{rgb}{0,0,0}
                \definecolor{c10}{rgb}{0,0,0}
                \definecolor{c11}{rgb}{0,0,0}
                \definecolor{c12}{rgb}{0,0,0}
                \definecolor{c13}{rgb}{0,0,0}
                \definecolor{c14}{rgb}{0,0,0}
                \definecolor{c15}{rgb}{0,0,0}
                \definecolor{c16}{rgb}{0,0,0}
        }%
        {}

} % this command creates some colours, based on the input. 



%%%             PACKAGE CUSTOMIZATION START             %%%


\usetikzlibrary{calc,patterns,shapes,arrows,positioning} %Cool libraries for drawing stuff

\sisetup{exponent-product=\cdot, output-product =\cdot,per-mode=fraction,range-phrase=--}%Nicer way to present units

%%%             PACKAGE CUSTOMIZATION END               %%%

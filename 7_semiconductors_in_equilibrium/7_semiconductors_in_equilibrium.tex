\documentclass[article,oneside]{memoir}

%cool packages
\usepackage[utf8]{inputenc}% That awesome utf8 characters
\usepackage[T1]{fontenc}% Those nice fonts
\usepackage{mathtools,amssymb,bm} %math
\usepackage{siunitx} % typesetting for units
\usepackage[draft,margin]{fixme} % for notes, remove draft if final version
\usepackage{graphicx} % for inserting figures
\usepackage{cleveref}
\usepackage{acronym} %for having acronyms
\usepackage{tikz} % for drawing stuff
\usepackage{pgfplots}[compat=1.16] %for plotting data or math functions
\usepackage{xcolor} % for having nicer colours
\usepackage{ifthen} % boolean checks

%penalties for having clubs or widows
\clubpenalty10000 
\widowpenalty10000

%cool memoir stuff
%The cool memoir stuff is wrapped in a check. This is done to see if the document class actually supports it.

%TEXINPUTS=./figures//pdf:%
\makeatletter%
\@ifclassloaded{memoir}%
  {%
\newsubfloat{figure} %allowing sub figures
\graphicspath{{./figures/}{../figures/}{../../figures/}{../figures/1/}{../figures/2/}{../figures/3/}{../figures/4/}{../figures/5/}{../figures/6/}{../figures/7/}{../figures/8/}{../figures/9/}{../figures/10/}} % no need for those pesky folder extensions in the actual file
  }%
  {}%
\makeatother%

% new commands

\newcommand{\foldfig}[2]{\includegraphics[#1]{#2/#2}}
\newcommand\numberthis{\addtocounter{equation}{1}\tag{\theequation}} %numbering equations in align* or aligned*

%\newcommand{colourscheme}[1]%
%{%
%        \ifthenelse{\eqal{#1}{1}}{%       
%                \definecolor{c1}{rgb}{0,0,0}
%                \definecolor{c2}{rgb}{0,0,0}
%                \definecolor{c3}{rgb}{0,0,0}
%                \definecolor{c4}{rgb}{0,0,0}
%                \definecolor{c5}{rgb}{0,0,0}
%                \definecolor{c6}{rgb}{0,0,0}
%                \definecolor{c7}{rgb}{0,0,0}
%                \definecolor{c8}{rgb}{0,0,0}
%                \definecolor{c9}{rgb}{0,0,0}
%                \definecolor{c10}{rgb}{0,0,0}
%                \definecolor{c11}{rgb}{0,0,0}
%                \definecolor{c12}{rgb}{0,0,0}
%                \definecolor{c13}{rgb}{0,0,0}
%                \definecolor{c14}{rgb}{0,0,0}
%                \definecolor{c15}{rgb}{0,0,0}
%                \definecolor{c16}{rgb}{0,0,0}
%        }%
%        {}%
%        \ifthenelse{\eqal{#1}{2}}{%       
%                \definecolor{c1}{rgb}{0,0,0}
%                \definecolor{c2}{rgb}{0,0,0}
%                \definecolor{c3}{rgb}{0,0,0}
%                \definecolor{c4}{rgb}{0,0,0}
%                \definecolor{c5}{rgb}{0,0,0}
%                \definecolor{c6}{rgb}{0,0,0}
%                \definecolor{c7}{rgb}{0,0,0}
%                \definecolor{c8}{rgb}{0,0,0}
%                \definecolor{c9}{rgb}{0,0,0}
%                \definecolor{c10}{rgb}{0,0,0}
%                \definecolor{c11}{rgb}{0,0,0}
%                \definecolor{c12}{rgb}{0,0,0}
%                \definecolor{c13}{rgb}{0,0,0}
%                \definecolor{c14}{rgb}{0,0,0}
%                \definecolor{c15}{rgb}{0,0,0}
%                \definecolor{c16}{rgb}{0,0,0}
%        }%
%        {}%
%} % this command creates some colours, based on the input. 

\definecolor{c1}{rgb}{1,1,1}
\definecolor{c2}{rgb}{0,0,0}
\definecolor{c3}{rgb}{1,0,1}
\definecolor{c4}{rgb}{1,0,1}
\definecolor{c5}{rgb}{1,0,0}
\definecolor{c6}{rgb}{0,1,0}
\definecolor{c7}{rgb}{0,0,1}
\definecolor{c8}{rgb}{0,0,0}
\definecolor{c9}{rgb}{0,0,0}
\definecolor{c10}{rgb}{0,0,0}
\definecolor{c11}{rgb}{0,0,0}
\definecolor{c12}{rgb}{0,0,0}
\definecolor{c13}{rgb}{0,0,0}
\definecolor{c14}{rgb}{0,0,0}
\definecolor{c15}{rgb}{0,0,0}
\definecolor{c16}{rgb}{0,0,0}

%%%             PACKAGE CUSTOMIZATION START             %%%


\usetikzlibrary{calc,patterns,shapes,arrows,positioning} %Cool libraries for drawing stuff

\sisetup{exponent-product=\cdot, output-product =\cdot,per-mode=fraction,range-phrase=--}%Nicer way to present units

%%%             PACKAGE CUSTOMIZATION END               %%%

\title{Semiconductors in equilibrium}
\author{Thorbjørn Erik Køppen Christensen}
\begin{document}
\chapter{Equilibrium distribution of electrons and holes}
The equilibrium distributions are:
\begin{align*}
        n(E)&=g_c(E)f_{F}(E)\\
        p(E)&=g_v(E)(1-f_F(E))\\
        \intertext{with $f_F$ being a Fermi--Dirac distribution:}
\end{align*}
$n$ is the electron distribution in the valence band and $g_c$ the density of states (DOS) in the conducting band, $p$ the hole distribution in the conducting band and $g_v$ the DOS in the valence band.

An intrinsic semiconductor is a perfect and perfectly pure semiconductor crystal. At $T=0k$ all states in the valence band is filled.

\foldfig{width=\columnwidth}{band_gap_schematic}
\chapter{Thermal equilibrium electron distribution}
assumptions: $E>E_{c}$, $E_c-E_F\gg kT$ then $E-E_F \gg kT$
\begin{align*}
        n_0&=\int_{E_c}^{E_{\textup{max}}}g_c(E)f_F(E)dE\\
        \intertext{$E_{\textup{max}}$ is the maximum possible energy, it can be assumed equal to $\infty$ (due to $f_F\xrightarrow[E\rightarrow\infty]{}0$)}
        f_{F}(E)&=\frac{1}{1+\exp\left( \frac{E-E_{F}}{kT} \right)}\\
        &\approx e^{-\frac{E-E_F}{kT}}\\
        n_0&=\int_{E_c}^{\infty}\frac{4\pi(2m_a^*)^{\frac{3}{2}}}{h^3}\sqrt{E-E_c}\exp\left( -\frac{E-E_F}{kT} \right)\,dE\qquad\textup{Let:}\quad\eta=\frac{E-E_c}{kT}\\
        &=\frac{4\pi\left( 2m_{n}^{*}kT \right)^{\frac{3}{2}}}{h^3}\exp\left( -\frac{E_c-E_F}{kT} \right)\underbrace{\int_0^{\infty}\eta^{\frac{1}{2}}e^{-\eta}\,d\eta }_{=\frac{\sqrt{\pi}}{2}}\\
        &=\underbrace{2\left( \frac{2\pi m_{n}^*kT}{h^2} \right)^{\frac{3}{2}}}_{N_c}e^{-\frac{E_c-E_F}{kT}}\\
        &=N_ce^{-\frac{E_c-E_F}{kT}}
\end{align*}
The same thing can be shown for holes, just swap $m_n^*$(effective mass of electron) with $m_p^*$(effective mass of hole) and $E-E_F$ with $E_F-E$ and $E_c-E_F$ with $E_F-E_v$.

$N_{c}$ is the \emph{effective density of states function in the conduction band}

For holes the constant is $N_v=2\left( \frac{2\pi m_{p}^*kT}{h^2} \right)^{\frac{3}{2}}$

\chapter{Intrinsic carrier concentration}
For the intrinsic system $E_{Fi}$ is the intrinsic Fermi energy.
\begin{align*}
        n_0=p_0=n_i=p_i&=N_ce^{-\frac{E_c-E_{Fi}}{kT}}\\
        n_i^2&=N_cN_ve^{-\frac{E_c-E_{Fi}}{kT}}e^{-\frac{E_{Fi}-E_{v}}{kT}}\\
        &=N_cN_ve^{-\frac{E_c-E_v}{Kt}}\\
        &=N_cN_ve^{-\frac{E_g}{Kt}}
\end{align*}
$E_g$ is the band gap.
\chapter{Intrinsic Fermi level}
\begin{align*}
        n_i&=N_ce^{-\frac{E_c-E_{Fi}}{kT}}\\
        &=N_ve^{-\frac{E_{Fi}-E_{v}}{kT}}\\
        E_{Fi}&=\frac{1}{2}\left( E_c+E_v \right) +\frac{1}{2}kT\ln\left( \frac{N_v}{N_c} \right)\\
        &=\frac{1}{2}\left( E_c+E_v \right) +\frac{3}{4}kT\ln\left( \frac{m_{p}^*}{m_n^*} \right)
\end{align*}
\chapter{Dopant atoms}
\foldfig{width=\columnwidth}{doner_acceptor}

Donors and acceptors change the properties, it also makes the semiconductor extrinsic instead of intrinsic.

\chapter{The extrinsic semiconductor}

\foldfig{width=\columnwidth}{extrinsic_semiconductor}

The donors/acceptors create a new equilibrium. This is gotten from adding and subtracting the intrinsic Fermi energy to the exponential term:
\begin{align*}
        n_0&=N_ce^{-\frac{(E_c-E_{Fi})+(E_F-E_{Fi})}{kT}}\\
        &=\underbrace{N_ce^{-\frac{(E_c-E_{Fi})}{kT}}}_{n_i}e^{\frac{E_F-E_{Fi}}{kT}}\\
        &=n_ie^{\frac{E_F-E_{Fi}}{kT}}\\
        \intertext{This leads to a new product at thermal equilibrium:}
        n_0p_0&=n_ie^{\frac{E_F-E_{Fi}}{kT}}n_ie^{\frac{E_{Fi}-E_{f}}{kT}}\\
        &=N_cN_ve^{-\frac{E_g}{kT}}\\
        &=n_i^2
\end{align*}
\chapter{Probability function}
\begin{align*}
        n_d&=\frac{N_d}{1+\frac{1}{2}\exp\left( \frac{E_d-E_F}{kT} \right)}\\
        \intertext{$n_d$ is the density of electrons occupying a donor level, $N_d$ the concentration of donors and $E_d$ the donor energy level. $\frac{1}{2}$ is from spin (also degeneracy factor)}
        &=N_d-N_d^+\\
        \intertext{$N_d^+$ is the concentration of ionized donors, now assume $(E_d-E_F)\gg kT$:}
        n_d&=\frac{N_d}{\frac{1}{2}\exp\left( \frac{E_d-E_F}{kT} \right)}\\
        &=2N_d\exp\left( -\frac{E_d-E_F}{kT} \right)\\
        n_0&=N_c\exp\left( -\frac{E_c-E_F}{kT} \right)\\
        \frac{n_d}{n_d+n_0}&=\frac{2N_d\exp\left( -\frac{E_d-E_F}{kT} \right)}{2N_d\exp\left( -\frac{E_d-E_F}{kT} \right)+N_c\exp\left( -\frac{E_c-E_F}{kT} \right)}\\
        &=\frac{1}{1+\frac{N_c}{2N_d}\exp\left( -\frac{E_c-E_d}{kT} \right)}
\end{align*}

\chapter{Thermal equilibrium electron concentration}
A compensated semiconductor has both acceptor and donor impurities
\begin{align*}
        n_0+N_a^-&=p_0+N_d^+\\
        n_0+(N_a-p_a)&=p_0+(N_d-n_d)\\
        \intertext{$n_d$ is the concentration of electrons in the donor states, $N_d^+$ the concentration of positively charged donor states, $p_a$ is the concentration of holes in the acceptor states, $N_a^-$ the concentration of negatively charged acceptor states}
        n_0+N_a&=p_0+N_d\\
        &=\frac{n_i^2}{n_0}+N_d\\
        0&= n_0^2-\left( N_d-N_a \right)n_0-n_i^2\\
        n_0&=\frac{N_d-N_a}{2}+\sqrt{\left( \frac{N_d-N_a}{2}^2+n_i^2 \right)}
\end{align*}
\chapter{Fermi level position}
\begin{align*}
        E_c-E_F&=kT\ln\left( \frac{N_c}{n_0} \right)\\
        &=kT\ln\left( \frac{N_c}{N_d} \right)\\
        \intertext{since}
        N_d&\gg n_i\\
        n_0&\simeq N_d
        \intertext{for an $n$ type semiconductor. To get the intrinsic Fermi level:}
        n_0&=n_ie^{\frac{E_F-E_{Fi}}{kT}}\\
        &\Downarrow\\
        E_{F}-E_{Fi}&=kT\ln\left( \frac{n_0}{n_i} \right)\\
\end{align*}
\end{document}
